\documentclass{ximera}

%\usepackage{todonotes}

\newcommand{\todo}{}

\usepackage{tkz-euclide}
\tikzset{>=stealth} %% cool arrow head
\tikzset{shorten <>/.style={ shorten >=#1, shorten <=#1 } } %% allows shorter vectors

\usetikzlibrary{backgrounds} %% for boxes around graphs
\usetikzlibrary{shapes,positioning}  %% Clouds and stars
\usetikzlibrary{matrix} %% for matrix
\usepgfplotslibrary{polar} %% for polar plots
\usetkzobj{all}
\usepackage[makeroom]{cancel} %% for strike outs
%\usepackage{mathtools} %% for pretty underbrace % Breaks Ximera
\usepackage{multicol}





\usepackage{array}
\setlength{\extrarowheight}{+.1cm}   
\newdimen\digitwidth
\settowidth\digitwidth{9}
\def\divrule#1#2{
\noalign{\moveright#1\digitwidth
\vbox{\hrule width#2\digitwidth}}}





\newcommand{\RR}{\mathbb R}
\newcommand{\R}{\mathbb R}
\newcommand{\N}{\mathbb N}
\newcommand{\Z}{\mathbb Z}

%\renewcommand{\d}{\,d\!}
\renewcommand{\d}{\mathop{}\!d}
\newcommand{\dd}[2][]{\frac{\d #1}{\d #2}}
\newcommand{\pp}[2][]{\frac{\partial #1}{\partial #2}}
\renewcommand{\l}{\ell}
\newcommand{\ddx}{\frac{d}{\d x}}

\newcommand{\zeroOverZero}{\ensuremath{\boldsymbol{\tfrac{0}{0}}}}
\newcommand{\inftyOverInfty}{\ensuremath{\boldsymbol{\tfrac{\infty}{\infty}}}}
\newcommand{\zeroOverInfty}{\ensuremath{\boldsymbol{\tfrac{0}{\infty}}}}
\newcommand{\zeroTimesInfty}{\ensuremath{\small\boldsymbol{0\cdot \infty}}}
\newcommand{\inftyMinusInfty}{\ensuremath{\small\boldsymbol{\infty - \infty}}}
\newcommand{\oneToInfty}{\ensuremath{\boldsymbol{1^\infty}}}
\newcommand{\zeroToZero}{\ensuremath{\boldsymbol{0^0}}}
\newcommand{\inftyToZero}{\ensuremath{\boldsymbol{\infty^0}}}



\newcommand{\numOverZero}{\ensuremath{\boldsymbol{\tfrac{\#}{0}}}}
\newcommand{\dfn}{\textbf}
%\newcommand{\unit}{\,\mathrm}
\newcommand{\unit}{\mathop{}\!\mathrm}
\newcommand{\eval}[1]{\bigg[ #1 \bigg]}
\newcommand{\seq}[1]{\left( #1 \right)}
\renewcommand{\epsilon}{\varepsilon}
\renewcommand{\iff}{\Leftrightarrow}

\DeclareMathOperator{\arccot}{arccot}
\DeclareMathOperator{\arcsec}{arcsec}
\DeclareMathOperator{\arccsc}{arccsc}
\DeclareMathOperator{\si}{Si}
\DeclareMathOperator{\proj}{proj}
\DeclareMathOperator{\scal}{scal}


\newcommand{\tightoverset}[2]{% for arrow vec
  \mathop{#2}\limits^{\vbox to -.5ex{\kern-0.75ex\hbox{$#1$}\vss}}}
\newcommand{\arrowvec}[1]{\tightoverset{\scriptstyle\rightharpoonup}{#1}}
\renewcommand{\vec}{\mathbf}
\newcommand{\veci}{\vec{i}}
\newcommand{\vecj}{\vec{j}}
\newcommand{\veck}{\vec{k}}
\newcommand{\vecl}{\boldsymbol{\l}}

\newcommand{\dotp}{\bullet}
\newcommand{\cross}{\boldsymbol\times}
\newcommand{\grad}{\boldsymbol\nabla}
\newcommand{\divergence}{\grad\dotp}
\newcommand{\curl}{\grad\cross}
%\DeclareMathOperator{\divergence}{divergence}
%\DeclareMathOperator{\curl}[1]{\grad\cross #1}


\colorlet{textColor}{black} 
\colorlet{background}{white}
\colorlet{penColor}{blue!50!black} % Color of a curve in a plot
\colorlet{penColor2}{red!50!black}% Color of a curve in a plot
\colorlet{penColor3}{red!50!blue} % Color of a curve in a plot
\colorlet{penColor4}{green!50!black} % Color of a curve in a plot
\colorlet{penColor5}{orange!80!black} % Color of a curve in a plot
\colorlet{fill1}{penColor!20} % Color of fill in a plot
\colorlet{fill2}{penColor2!20} % Color of fill in a plot
\colorlet{fillp}{fill1} % Color of positive area
\colorlet{filln}{penColor2!20} % Color of negative area
\colorlet{fill3}{penColor3!20} % Fill
\colorlet{fill4}{penColor4!20} % Fill
\colorlet{fill5}{penColor5!20} % Fill
\colorlet{gridColor}{gray!50} % Color of grid in a plot

\newcommand{\surfaceColor}{violet}
\newcommand{\surfaceColorTwo}{redyellow}
\newcommand{\sliceColor}{greenyellow}




\pgfmathdeclarefunction{gauss}{2}{% gives gaussian
  \pgfmathparse{1/(#2*sqrt(2*pi))*exp(-((x-#1)^2)/(2*#2^2))}%
}


%%%%%%%%%%%%%
%% Vectors
%%%%%%%%%%%%%

%% Simple horiz vectors
\renewcommand{\vector}[1]{\left\langle #1\right\rangle}


%% %% Complex Horiz Vectors with angle brackets
%% \makeatletter
%% \renewcommand{\vector}[2][ , ]{\left\langle%
%%   \def\nextitem{\def\nextitem{#1}}%
%%   \@for \el:=#2\do{\nextitem\el}\right\rangle%
%% }
%% \makeatother

%% %% Vertical Vectors
%% \def\vector#1{\begin{bmatrix}\vecListA#1,,\end{bmatrix}}
%% \def\vecListA#1,{\if,#1,\else #1\cr \expandafter \vecListA \fi}

%%%%%%%%%%%%%
%% End of vectors
%%%%%%%%%%%%%

%\newcommand{\fullwidth}{}
%\newcommand{\normalwidth}{}



%% makes a snazzy t-chart for evaluating functions
%\newenvironment{tchart}{\rowcolors{2}{}{background!90!textColor}\array}{\endarray}

%%This is to help with formatting on future title pages.
\newenvironment{sectionOutcomes}{}{} 



%% Flowchart stuff
%\tikzstyle{startstop} = [rectangle, rounded corners, minimum width=3cm, minimum height=1cm,text centered, draw=black]
%\tikzstyle{question} = [rectangle, minimum width=3cm, minimum height=1cm, text centered, draw=black]
%\tikzstyle{decision} = [trapezium, trapezium left angle=70, trapezium right angle=110, minimum width=3cm, minimum height=1cm, text centered, draw=black]
%\tikzstyle{question} = [rectangle, rounded corners, minimum width=3cm, minimum height=1cm,text centered, draw=black]
%\tikzstyle{process} = [rectangle, minimum width=3cm, minimum height=1cm, text centered, draw=black]
%\tikzstyle{decision} = [trapezium, trapezium left angle=70, trapezium right angle=110, minimum width=3cm, minimum height=1cm, text centered, draw=black]


\outcome{Find the domain and range of a function.}
\outcome{Distinguish between functions by considering their domains.}
\outcome{Perform basic operations and compositions on functions.}
\outcome{Work with piecewise defined functions.}
\outcome{Recognize different representations of the same function.}

\title[Dig-In:]{Compositions of functions}
\begin{document}
\begin{abstract}
  We discuss compositions of functions.
\end{abstract}
\maketitle


Given two functions, we can compose them. Let's give an example in a
``real context.''

\begin{example}
  Let
  \[
  g(m) = \text{the amount of gas one can buy with $m$ dollars,}
  \]
  and let
  \[
  f(g) = \text{how far one can drive with $g$ gallons of gas.}
  \]
  What does $f(g(m))$ represent in this setting?
  \begin{explanation}
    With $f(g(m))$ we first relate how far one can drive with
    $\answer[given]{g}$ gallons of gas, and this in turn is determined
    by how much money $\answer[given]{m}$ one has. Hence $f(g(m))$ represents how far
    one can drive with $\answer[given]{m}$ dollars.
  \end{explanation}
\end{example}

Composition of functions can be thought of as putting one function
inside another.  We use the notation
\[
(f\circ g)(x) = f(g(x)).
\]
\begin{warning}
  The composition $f\circ g$ only makes sense if
  \[
  \{\text{the range of $g$}\}
  \text{ is contained in or equal to }
  \{\text{the domain of $f$}\}
  \]
\end{warning}

\begin{example}
 Suppose we have
\begin{align*}
  f(x)&={{x}^{2}}+5x+4 &&\text{for $-\infty< x< \infty$,}\\
  g(x)&= x+7 &&\text{for $-\infty< x< \infty$.}\\
\end{align*}
Find $f(g(x))$ and state its domain.
\begin{explanation}
  The range of $g$ is $-\infty< x< \infty$, which is equal to the
  domain of $f$. This means the domain of $f\circ g$ is $-\infty< x<
  \infty$. Next, we substitute $x+7$ for each instance of $\answer[given]{x}$ found
  in
  \[
  f(x)={{x}^{2}}+5x+4
  \]
  and so
  \begin{align*}
  f(g(x)) &=f(x+7)\\
  &=\answer[given]{{{(x+7)}^{2}}+5(x+7)+4}.
  \end{align*}
\end{explanation}
\end{example}

Now let's try an example with a more restricted domain.

\begin{example}
 Suppose we have:
\begin{align*}
  f(x)&=x^2 &&\text{for $-\infty< x< \infty$,}\\
  g(x)&= \sqrt{x} &&\text{for $0\le x< \infty$.}\\
\end{align*}
Find $f(g(x))$ and state its domain.
\begin{explanation}
  The domain of $g$ is $0\le x< \infty$. From this we can see that the
  range of $g$ is $\answer[given]{0}\le x< \infty$. This is contained
  in the domain of $f$.

  This means that the domain of $f\circ g$ is $0\le x< \infty$.  Next,
  we substitute $\answer[given]{\sqrt{x}}$ for each instance of $x$
  found in
  \[
  f(x)={{x}^{2}}
  \]
  and so
  \begin{align*}
  f(g(x))&=f(\sqrt{x})\\
  &=\left(\sqrt{x}\right)^2.
  \end{align*}
  We can summarize our results as a piecewise function, which
  looks somewhat interesting:
  \[
  (f\circ g)(x) = 
  \begin{cases}
    x & \text{if $0\le x < \infty$}\\
   \text{undefined} &\text{otherwise}. 
  \end{cases}
  \]
\end{explanation}
\end{example}


\begin{example}
 Suppose we have:
\begin{align*}
  f(x)&=\sqrt{x} &&\text{for $0\le x< \infty$,}\\
  g(x)&= x^2 &&\text{for $-\infty< x< \infty$.}
\end{align*}
Find $f(g(x))$ and state its domain.
\begin{explanation}
  While the domain of $g$ is $-\infty< x< \infty$, its range is only
  $0 \le x<\infty$. This is exactly the domain of $f$. This means that
  the domain of $f\circ g$ is $-\infty< x< \infty$. %%BADBAD Explain more
  Now we may substitute $\answer[given]{x^2}$ for each instance of
  $\answer[given]{x}$ found in
  \[
  f(x)=\sqrt{x}
  \]
  and so
  \begin{align*}
  f(g(x))&=f(x^2)\\
  &=\sqrt{x^2},\\
  &=|x|.
  \end{align*}
\end{explanation}
\end{example}

Compare and contrast the previous two examples.  We used the same
functions for each example, but composed them in different ways.  The resulting
compositions are not only different, they have different domains!



\end{document}
